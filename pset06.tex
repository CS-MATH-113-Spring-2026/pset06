\documentclass[a4paper]{exam}

\usepackage{amsmath,amssymb, amsthm}
\usepackage{geometry}
\usepackage{graphicx}
\usepackage{hyperref}
\usepackage{titling}



% Header and footer.
\pagestyle{headandfoot}
\runningheadrule
\runningfootrule
\runningheader{CS/MATH 113, SPRING 2026}{Pset 06: Proofs, Sets and Functions}{\theauthor}
\runningfooter{}{Page \thepage\ of \numpages}{}
\firstpageheader{}{}{}

% \printanswers %Uncomment this line

\title{Problem Set 06: Proofs, Sets and Functions}
\author{Blingblong} % <=== replace with your student ID, e.g. xy012345
\date{CS/MATH 113 Discrete Mathematics\\Habib University\\Spring 2026}

\boxedpoints

\begin{document}
\maketitle

\section*{Problems}

\begin{questions}

    \question Prove or disprove the following claim: there exists irrational numbers $a$ and $b$ such that $a^b$ is rational.
    \begin{solution}
        % Enter solution here 
    \end{solution}

    \question Show that there are infinitely many primes, in other words the set containing all prime numbers is infinite.
    \begin{solution}
        % Enter solution here 
    \end{solution}

    \question Show that $\emptyset$, $\{\emptyset\}$, $\{\{\emptyset\}\}$, and $\{\emptyset, \{\emptyset\}\}$ are all different sets.
    \begin{solution}
        % Enter solution here        
    \end{solution}

    \question Among the set identities you have studied are the Absorption laws (see page 136 of book). Prove that these identities hold, i.e.
    \begin{parts}
        \part $A \cup ( A \cap B) = A$
        \begin{solution}
            % Enter solution here
        \end{solution}

        \part $A \cap ( A \cup B) = A$
        \begin{solution}
            % Enter solution here
        \end{solution}
    \end{parts}

    \question Let $A$ and $B$ be subsets of a universal set $U$. 
    \begin{parts}
        \part Show that $A \subseteq B$ if and only if $\overline{B} \subseteq \overline{A}$.
        \begin{solution}
            % Enter solution here
        \end{solution} 

        \part Show that for all sets $A$ and $B$, $(A \cap B) \cup (A \cap \overline{B}) = A$.
        \begin{solution}
            % Enter solution here
        \end{solution} 
    \end{parts}

    \question Let $A$ and $B$ be some sets. Show that $\mathcal{P}(A) \subseteq \mathcal{P}(B) \iff A \subseteq B$.
    \begin{solution}
        % Enter solution here
    \end{solution}

    \question Let $A$, $B$, $C$, and $D$ be some nonempty sets. Show that $A \times B \subseteq C \times D \iff A \subseteq C \text{ and } B \subseteq D$. Furthermore, show that $ A \times B = C \times D \iff A = C \text{ and } B = D$. \\ 
    What happens if the restriction about them being nonempty sets is removed? Does the result still hold?
    \begin{solution}
        % Enter solution here
    \end{solution}


    \question The symmetric difference of $A$ and $B$, denoted by $A \oplus B$, is the set containing those elements in either $A$ or $B$, but not in both $A$ and $B$. \\
    \begin{parts}
        \part Define $A \oplus B$ in terms of elementary set operations that we have studied in class.
        \begin{solution}
            % Enter solution here
        \end{solution} 

        \part Let $A$, $B$, and $C$ be some sets. Prove or disprove the following: If $A \oplus C = B \oplus C$ then $A = B$.
        \begin{solution}
            % Enter solution here
        \end{solution} 
    \end{parts}


    \question Let $S$ be the set that contains a set $x$ if the set $x$ does not belong to itself, so that $S = \{\, x \mid x \notin x \,\}$. 
    \begin{parts}
        \part Show the assumption that $S$ is a member of $S$ leads to a contradiction.
        \begin{solution}
            % Enter solution here
        \end{solution}

        \part Show the assumption that $S$ is not a member of $S$ leads to a contradiction.
        \begin{solution}
            % Enter solution here
        \end{solution}
        
        \part By parts (a) and (b) it follows that the set S cannot be defined as it was. What you have just shown is called the \textbf{Russell's paradox}. There are various ways modern set theories avoid this paradox. 

        How do you think this paradox is avoided in modern set theory? Suggest an axiom we can add in set theory to avoid this paradox.
        \begin{solution}
            % Enter solution here
        \end{solution}
    \end{parts}
    
    
    
    
    

    \question Let $X$, $Y$, $Z$ be some sets and let $f: X \to Y$, $f': X \to Y$, $g: Y \to Z$, and $g': Y \to Z$ be functions.
    \begin{parts}
        \part Show that if $g \circ f$ is injective, then $f$ must be injective. Is it true that $g$ must also be injective? 
        \begin{solution}
            % Enter solution here
        \end{solution}

        \part Show that if $g \circ f$ is surjective, then $g$ must be surjective. Is it true that $f$ must also be surjective?
        \begin{solution}
            % Enter solution here
        \end{solution}

        \part Show that if $g \circ f = g \circ f'$ and $g$ is an injection then $f = f'$.
        \begin{solution}
            % Enter solution here
        \end{solution}

        \part Show that if $g' \circ f = g \circ f$ and $f$ is a surjection then $g = g'$.
        \begin{solution}
            % Enter solution here
        \end{solution}

        \part Show that if $f$ and $g$ are bijective, then so is $g \circ f$, and we have $(g \circ f)^{-1} = f^{-1} \circ g^{-1}$.
        \begin{solution}
            % Enter solution here
        \end{solution}

    \end{parts}

    
      
\end{questions}
\end{document}

%%% Local Variables:
%%% mode: latex
%%% TeX-master: t
%%% End:
